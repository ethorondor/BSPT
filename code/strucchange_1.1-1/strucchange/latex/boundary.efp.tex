\Header{boundary.efp}{Boundary for Empirical Fluctuation Processes}
\keyword{regression}{boundary.efp}
\begin{Description}\relax
Computes boundary for an object of class \code{"efp"}\end{Description}
\begin{Usage}
\begin{verbatim}
## S3 method for class 'efp':
boundary(x, alpha = 0.05, alt.boundary = FALSE,
   functional = "max", ...)
\end{verbatim}
\end{Usage}
\begin{Arguments}
\begin{ldescription}
\item[\code{x}] an object of class \code{"efp"}.
\item[\code{alpha}] numeric from interval (0,1) indicating the confidence level for
which the boundary of the corresponding test will be computed.
\item[\code{alt.boundary}] logical. If set to \code{TRUE} alternative boundaries
(instead of the standard linear boundaries) will be computed (for Brownian
bridge type processes only).
\item[\code{functional}] indicates which functional should be applied to the
empirical fluctuation process. See also \code{\Link{plot.efp}}.
\item[\code{...}] currently not used.
\end{ldescription}
\end{Arguments}
\begin{Value}
an object of class \code{"ts"} with the same time properties as
the process in \code{x}\end{Value}
\begin{SeeAlso}\relax
\code{\Link{efp}}, \code{\Link{plot.efp}}\end{SeeAlso}
\begin{Examples}
\begin{ExampleCode}
## Load dataset "nhtemp" with average yearly temperatures in New Haven
data(nhtemp)
## plot the data
plot(nhtemp)

## test the model null hypothesis that the average temperature remains constant
## over the years
## compute OLS-CUSUM fluctuation process
temp.cus <- efp(nhtemp ~ 1, type = "OLS-CUSUM")
## plot the process without boundaries
plot(temp.cus, alpha = 0.01, boundary = FALSE)
## add the boundaries in another colour
bound <- boundary(temp.cus, alpha = 0.01)
lines(bound, col=4)
lines(-bound, col=4)
\end{ExampleCode}
\end{Examples}

