\Header{recresid}{Recursive Residuals}
\methalias{recresid.default}{recresid}
\methalias{recresid.formula}{recresid}
\methalias{recresid.lm}{recresid}
\keyword{regression}{recresid}
\begin{Description}\relax
A generic function for computing the recursive residuals
(standardized one step prediction errors) of a linear regression model.\end{Description}
\begin{Usage}
\begin{verbatim}
## Default S3 method:
recresid(x, y, ...)
## S3 method for class 'formula':
recresid(formula, data = list(), ...)
## S3 method for class 'lm':
recresid(x, data = list(), ...)
\end{verbatim}
\end{Usage}
\begin{Arguments}
\begin{ldescription}
\item[\code{x, y, formula}] specification of the linear regression model:
either by a regressor matrix \code{x} and a response variable \code{y},
or by a \code{formula} or by a fitted object \code{x} of class \code{"lm"}.
\item[\code{data}] an optional data frame containing the variables in the model. By
default the variables are taken from the environment which \code{recresid} is
called from. Specifying \code{data} might also be necessary when applying
\code{recresid} to a fitted model of class \code{"lm"} if this does not
contain the regressor matrix and the response.
\item[\code{...}] \emph{currently not used.}
\end{ldescription}
\end{Arguments}
\begin{Details}\relax
Under the usual assumptions for the linear regression model the
recdursive residuals are (asymptotically) normal and
i.i.d. (see Brown, Durbin, Evans (1975) for details).\end{Details}
\begin{Value}
A vector containing the recursive residuals.\end{Value}
\begin{References}\relax
Brown R.L., Durbin J., Evans J.M. (1975), Techniques for
testing constancy of regression relationships over time, \emph{Journal of the
Royal Statistal Society}, B, \bold{37}, 149-163.\end{References}
\begin{SeeAlso}\relax
\code{\Link{efp}}\end{SeeAlso}
\begin{Examples}
\begin{ExampleCode}
x <- rnorm(100)
x[51:100] <- x[51:100] + 2
rr <- recresid(x ~ 1)
plot(cumsum(rr), type = "l")

plot(efp(x ~ 1, type = "Rec-CUSUM"))
\end{ExampleCode}
\end{Examples}

