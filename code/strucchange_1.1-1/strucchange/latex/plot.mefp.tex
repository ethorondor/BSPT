\Header{plot.mefp}{Plot Methods for mefp Objects}
\alias{lines.mefp}{plot.mefp}
\keyword{hplot}{plot.mefp}
\begin{Description}\relax
This is a method of the generic \code{\Link{plot}} function for
for \code{"mefp"} objects as returned by \code{\Link{mefp}} or
\code{\Link{monitor}}. It plots the emprical fluctuation process (or a
functional therof) as a time series plot, and includes boundaries
corresponding to the significance level of the monitoring procedure.\end{Description}
\begin{Usage}
\begin{verbatim}
## S3 method for class 'mefp':
plot(x, boundary = TRUE, functional = "max", main = NULL,
    ylab = "empirical fluctuation process", ylim = NULL, ...)
\end{verbatim}
\end{Usage}
\begin{Arguments}
\begin{ldescription}
\item[\code{x}] an object of class \code{"mefp"}.
\item[\code{boundary}] if \code{FALSE}, plotting of boundaries is suppressed.
\item[\code{functional}] indicates which functional should be applied to a
multivariate empirical process. If set to \code{NULL} all dimensions
of the process (one process per coefficient in the linear model) are
plotted. 
\item[\code{main, ylab, ylim, ...}] high-level \code{\Link{plot}} function parameters.
\end{ldescription}
\end{Arguments}
\begin{SeeAlso}\relax
\code{\Link{mefp}}\end{SeeAlso}
\begin{Examples}
\begin{ExampleCode}
df1 <- data.frame(y=rnorm(300))
df1[150:300,"y"] <- df1[150:300,"y"]+1
me1 <- mefp(y~1, data=df1[1:50,,drop=FALSE], type="ME", h=1,
              alpha=0.05)
me2 <- monitor(me1, data=df1)

plot(me2)
\end{ExampleCode}
\end{Examples}

