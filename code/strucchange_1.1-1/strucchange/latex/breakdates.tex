\Header{breakdates}{Breakdates Corresponding to Breakpoints}
\methalias{breakdates.breakpoints}{breakdates}
\alias{breakdates.confint.breakpoints}{breakdates}
\keyword{regression}{breakdates}
\begin{Description}\relax
A generic function for computing the breakdates corresponding
to breakpoints (and their confidence intervals).\end{Description}
\begin{Usage}
\begin{verbatim}
breakdates(obj, format.times = FALSE, ...)
\end{verbatim}
\end{Usage}
\begin{Arguments}
\begin{ldescription}
\item[\code{obj}] An object of class \code{"breakpoints"}, \code{"breakpointsfull"} or their
confidence intervals as returned by \code{\Link{confint}}.
\item[\code{format.times}] logical. If set to \code{TRUE} a vector of
strings with the formatted breakdates. See details for more
information.
\item[\code{...}] currently not used.
\end{ldescription}
\end{Arguments}
\begin{Details}\relax
Breakpoints are the number of observations that are the last in one
segment and breakdates are the corresponding points on the underlying
time scale. The breakdates can be formatted which enhances readability
in particular for quarterly or monthly time series. For example the
breakdate \code{2002.75} of a monthly time series will be formatted to
\code{"2002(10)"}.\end{Details}
\begin{Value}
A vector or matrix containing the breakdates.\end{Value}
\begin{SeeAlso}\relax
\code{\Link{breakpoints}}, \code{\Link{confint}}\end{SeeAlso}
\begin{Examples}
\begin{ExampleCode}
require(ts)

## Nile data with one breakpoint: the annual flows drop in 1898
## because the first Ashwan dam was built
data(Nile)
plot(Nile)

bp.nile <- breakpoints(Nile ~ 1)
summary(bp.nile)
plot(bp.nile)

## compute breakdates corresponding to the
## breakpoints of minimum BIC segmentation
breakdates(bp.nile)

## confidence intervals
ci.nile <- confint(bp.nile)
breakdates(ci.nile)
ci.nile

plot(Nile)
lines(ci.nile)
\end{ExampleCode}
\end{Examples}

