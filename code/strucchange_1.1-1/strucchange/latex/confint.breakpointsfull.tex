\Header{confint.breakpointsfull}{Confidence Intervals for Breakpoints}
\alias{lines.confint.breakpoints}{confint.breakpointsfull}
\alias{print.confint.breakpoints}{confint.breakpointsfull}
\keyword{regression}{confint.breakpointsfull}
\begin{Description}\relax
Computes confidence intervals for breakpoints.\end{Description}
\begin{Usage}
\begin{verbatim}
## S3 method for class 'breakpointsfull':
confint(object, parm = NULL, level = 0.95,
    breaks = NULL, het.reg = TRUE, het.err = TRUE, ...)
## S3 method for class 'confint.breakpoints':
lines(x, col = 2, angle = 90, length = 0.05,
    code = 3, at = NULL, breakpoints = TRUE, ...)
\end{verbatim}
\end{Usage}
\begin{Arguments}
\begin{ldescription}
\item[\code{object}] an object of class \code{"breakpointsfull"} as computed by
\code{\Link{breakpoints}} from a \code{formula}.
\item[\code{parm}] the same as \code{breaks}, only one of the two should be
specified.
\item[\code{level}] the confidence level required.
\item[\code{breaks}] an integer specifying the number of breaks to be used.
By default the breaks of the minimum BIC partition are used.
\item[\code{het.reg}] logical. Should heterogenous regressors be assumed? If set
to \code{FALSE} the distribution of the regressors is assumed to be
homogenous over the segments.
\item[\code{het.err}] logical. Should heterogenous errors be assumed? If set
to \code{FALSE} the distribution of the errors is assumed to be
homogenous over the segments.
\item[\code{x}] an object of class \code{"confint.breakpoints"} as returned by
\code{confint}.
\item[\code{col, angle, length, code}] arguments passed to \code{\Link{arrows}}.
\item[\code{at}] position on the y axis, where the confidence arrows should be
drawn. By default they are drawn at the bottom of the plot.
\item[\code{breakpoints}] logical. If \code{TRUE} vertical lines for the breakpoints
are drawn.
\item[\code{...}] \emph{currently not used}.
\end{ldescription}
\end{Arguments}
\begin{Details}\relax
As the breakpoints are integers (observation numbers) the corresponding
confidence intervals are also rounded to integers.

The distribution function used for the computation of confidence
intervals of breakpoints is given in Bai (1997). The procedure, in
particular the usage of heterogenous regressors and/or errors, is
described in more detail in Bai \& Perron (2003).

The breakpoints should be computed from a formula with \code{breakpoints},
then the confidence intervals for the breakpoints can be derived by
\code{confint} and these can be visualized by \code{lines}. For an
example see below.\end{Details}
\begin{Value}
A matrix containing the breakpoints and their lower and upper
confidence boundary for the given level.\end{Value}
\begin{References}\relax
Bai J. (1997), Estimation of a Change Point in Multiple Regression Models,
\emph{Review of Economics and Statistics}, \bold{79}, 551-563.

Bai J., Perron P. (2003), Computation and Analysis of Multiple Structural Change
Models, \emph{Journal of Applied Econometrics}, \bold{18}, 1-22.\end{References}
\begin{SeeAlso}\relax
\code{\Link{breakpoints}}\end{SeeAlso}
\begin{Examples}
\begin{ExampleCode}
require(ts)

## Nile data with one breakpoint: the annual flows drop in 1898
## because the first Ashwan dam was built
data(Nile)
plot(Nile)

## dating breaks
bp.nile <- breakpoints(Nile ~ 1)
ci.nile <- confint(bp.nile, breaks = 1)
lines(ci.nile)
\end{ExampleCode}
\end{Examples}

