\Header{breakfactor}{Factor Coding of Segmentations}
\keyword{regression}{breakfactor}
\begin{Description}\relax
Generates a factor encoding the segmentation given by
a set of breakpoints.\end{Description}
\begin{Usage}
\begin{verbatim}
breakfactor(obj, breaks = NULL, labels = NULL, ...)
\end{verbatim}
\end{Usage}
\begin{Arguments}
\begin{ldescription}
\item[\code{obj}] An object of class \code{"breakpoints"} or
\code{"breakpointsfull"} respectively.
\item[\code{breaks}] an integer specifying the number of breaks
to extract (only if \code{obj} is of class \code{"breakpointsfull"}),
by default the minimum BIC partition is used.
\item[\code{labels}] a vector of labels for the returned factor,
by default the segments are numbered starting from
\code{"segment1"}.
\item[\code{...}] further arguments passed to \code{factor}.
\end{ldescription}
\end{Arguments}
\begin{Value}
A factor encoding the segmentation.\end{Value}
\begin{SeeAlso}\relax
\code{\Link{breakpoints}}\end{SeeAlso}
\begin{Examples}
\begin{ExampleCode}
require(ts)

## Nile data with one breakpoint: the annual flows drop in 1898
## because the first Ashwan dam was built
data(Nile)
plot(Nile)

## compute breakpoints
bp.nile <- breakpoints(Nile ~ 1)

## fit and visualize segmented and unsegmented model
fm0 <- lm(Nile ~ 1)
fm1 <- lm(Nile ~ breakfactor(bp.nile, breaks = 1))

lines(fitted(fm0), col = 3)
lines(fitted(fm1), col = 4)
lines(bp.nile, breaks = 1)
\end{ExampleCode}
\end{Examples}

