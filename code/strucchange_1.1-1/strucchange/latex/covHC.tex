\Header{covHC}{Heteroskedasticity-Consistent Covariance Matrix Estimation}
\keyword{htest}{covHC}
\begin{Description}\relax
Heteroskedasticity-consistent estimation of the covariance matrix of the
coefficient estimates in a linear regression model.\end{Description}
\begin{Usage}
\begin{verbatim}
covHC(formula, type = c("HC2", "const", "HC", "HC1", "HC3"),
   data=list())
\end{verbatim}
\end{Usage}
\begin{Arguments}
\begin{ldescription}
\item[\code{formula}] a symbolic description for the model to be fitted.
\item[\code{type}] a character string specifying the estimation type. For
details see below.
\item[\code{data}] an optional data frame containing the variables in the model.
By default the variables are taken from the environment which \code{covHC}
is called from.
\end{ldescription}
\end{Arguments}
\begin{Details}\relax
When \code{type = "const"} constant variances are assumed and
and \code{covHC} gives the usual estimate of the covariance matrix of
the coefficient estimates:

\deqn{\hat \sigma^2 (X^\top X)^{-1}}{sigma^2 (X'X)^{-1}}

All other methods do not assume constant variances and are suitable in case of
heteroskedasticity. \code{"HC"} gives White's estimator; for details see the
references.\end{Details}
\begin{Value}
A matrix containing the covariance matrix estimate.\end{Value}
\begin{References}\relax
MacKinnon J. G., White H. (1985),
Some heteroskedasticity-consistent
covariance matrix estimators with improved finite sample properties.
\emph{Journal of Econometrics} \bold{29}, 305-325\end{References}
\begin{SeeAlso}\relax
\code{\Link{lm}}\end{SeeAlso}
\begin{Examples}
\begin{ExampleCode}
## generate linear regression relationship
## with homoskedastic variances
x <- sin(1:100)
y <- 1 + x + rnorm(100)
## compute usual covariance matrix of coefficient estimates
covHC(y~x, type="const")

sigma2 <- sum(residuals(lm(y~x))^2)/98
sigma2 * solve.crossprod(cbind(1,x))
\end{ExampleCode}
\end{Examples}

