\Header{boundary.Fstats}{Boundary for F Statistics}
\keyword{regression}{boundary.Fstats}
\begin{Description}\relax
Computes boundary for an object of class \code{"Fstats"}\end{Description}
\begin{Usage}
\begin{verbatim}
## S3 method for class 'Fstats':
boundary(x, alpha = 0.05, pval = FALSE, aveF = FALSE,
    asymptotic = FALSE, ...)\end{verbatim}
\end{Usage}
\begin{Arguments}
\begin{ldescription}
\item[\code{x}] an object of class \code{"Fstats"}.
\item[\code{alpha}] numeric from interval (0,1) indicating the confidence level for
which the boundary of the supF test will be computed.
\item[\code{pval}] logical. If set to \code{TRUE} a boundary for the corresponding p
values will be computed.
\item[\code{aveF}] logical. If set to \code{TRUE} the boundary of the aveF (instead of
the supF) test will be computed. The resulting boundary then is a boundary
for the mean of the F statistics rather than for the F statistics
themselves.
\item[\code{asymptotic}] logical. If set to \code{TRUE} the asymptotic (chi-square)
distribution instead of the exact (F) distribution will be used to compute
the p values (only if \code{pval} is \code{TRUE}).
\item[\code{...}] currently not used.
\end{ldescription}
\end{Arguments}
\begin{Value}
an object of class \code{"ts"} with the same time properties as
the time series in \code{x}\end{Value}
\begin{SeeAlso}\relax
\code{\Link{Fstats}}, \code{\Link{plot.Fstats}}\end{SeeAlso}
\begin{Examples}
\begin{ExampleCode}
## Load dataset "nhtemp" with average yearly temperatures in New Haven
data(nhtemp)
## plot the data
plot(nhtemp)

## test the model null hypothesis that the average temperature remains
## constant over the years for potential break points between 1941
## (corresponds to from = 0.5) and 1962 (corresponds to to = 0.85)
## compute F statistics
fs <- Fstats(nhtemp ~ 1, from = 0.5, to = 0.85)
## plot the p values without boundary
plot(fs, pval = TRUE, alpha = 0.01)
## add the boundary in another colour
lines(boundary(fs, pval = TRUE, alpha = 0.01), col = 2)
\end{ExampleCode}
\end{Examples}

