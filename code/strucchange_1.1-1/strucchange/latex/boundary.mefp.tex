\Header{boundary.mefp}{Boundary Function for Monitoring of Structural Changes}
\keyword{regression}{boundary.mefp}
\begin{Description}\relax
Computes boundary for an object of class \code{"mefp"}\end{Description}
\begin{Usage}
\begin{verbatim}
## S3 method for class 'mefp':
boundary(x, ...)\end{verbatim}
\end{Usage}
\begin{Arguments}
\begin{ldescription}
\item[\code{x}] an object of class \code{"mefp"}.
\item[\code{...}] currently not used.
\end{ldescription}
\end{Arguments}
\begin{Value}
an object of class \code{"ts"} with the same time properties as
the monitored process\end{Value}
\begin{SeeAlso}\relax
\code{\Link{mefp}}, \code{\Link{plot.mefp}}\end{SeeAlso}
\begin{Examples}
\begin{ExampleCode}
df1 <- data.frame(y=rnorm(300))
df1[150:300,"y"] <- df1[150:300,"y"]+1
me1 <- mefp(y~1, data=df1[1:50,,drop=FALSE], type="ME", h=1,
              alpha=0.05)
me2 <- monitor(me1, data=df1)

plot(me2, boundary=FALSE)
lines(boundary(me2), col="green", lty="44")
\end{ExampleCode}
\end{Examples}

