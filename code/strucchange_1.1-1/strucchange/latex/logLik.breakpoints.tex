\Header{logLik.breakpoints}{Log Likelihood and Information Criteria for Breakpoints}
\alias{AIC.breakpointsfull}{logLik.breakpoints}
\alias{logLik.breakpointsfull}{logLik.breakpoints}
\keyword{regression}{logLik.breakpoints}
\begin{Description}\relax
Computation of log likelihood and AIC type information criteria
for partitions given by breakpoints.\end{Description}
\begin{Usage}
\begin{verbatim}
## S3 method for class 'breakpointsfull':
logLik(object, breaks = NULL, ...)
## S3 method for class 'breakpointsfull':
AIC(object, breaks = NULL, ..., k = 2)
\end{verbatim}
\end{Usage}
\begin{Arguments}
\begin{ldescription}
\item[\code{object}] an object of class \code{"breakpoints"} or \code{"breakpointsfull"}.
\item[\code{breaks}] if \code{object} is of class \code{"breakpointsfull"} the
number of breaks can be specified.
\item[\code{...}] \emph{currently not used}.
\item[\code{k}] the penalty parameter to be used, the default \code{k = 2}
is the classical AIC, \code{k = log(n)} gives the BIC, if \code{n}
is the number of observations.
\end{ldescription}
\end{Arguments}
\begin{Details}\relax
As for linear models the log likelihood is computed on a normal model and
the degrees of freedom are the number of regression coefficients multiplied
by the number of segements plus the number of estimated breakpoints plus
1 for the error variance.

If \code{AIC} is applied to an object of class \code{"breakpointsfull"}
\code{breaks} can be a vector of integers and the AIC for each corresponding
partition will be returned. By default the maximal number of breaks stored
in the \code{object} is used. See below for an example.\end{Details}
\begin{Value}
An object of class \code{"logLik"} or a simple vector containing
the AIC respectively.\end{Value}
\begin{SeeAlso}\relax
\code{\Link{breakpoints}}\end{SeeAlso}
\begin{Examples}
\begin{ExampleCode}
require(ts)

## Nile data with one breakpoint: the annual flows drop in 1898
## because the first Ashwan dam was built
data(Nile)
plot(Nile)

bp.nile <- breakpoints(Nile ~ 1)
summary(bp.nile)
plot(bp.nile)

## BIC of partitions with0 to 5 breakpoints
plot(0:5, AIC(bp.nile, k = log(bp.nile$nobs)), type = "b")
## AIC
plot(0:5, AIC(bp.nile), type = "b")

## BIC, AIC, log likelihood of a single partition
bp.nile1 <- breakpoints(bp.nile, breaks = 1)
AIC(bp.nile1, k = log(bp.nile1$nobs))
AIC(bp.nile1)
logLik(bp.nile1)
\end{ExampleCode}
\end{Examples}

