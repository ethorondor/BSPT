\Header{sctest.efp}{Generalized Fluctuation Tests}
\keyword{htest}{sctest.efp}
\begin{Description}\relax
Performs a generalized fluctuation test.\end{Description}
\begin{Usage}
\begin{verbatim}
## S3 method for class 'efp':
sctest(x, alt.boundary = FALSE,
    functional = c("max", "range", "maxL2", "meanL2"), ...)
\end{verbatim}
\end{Usage}
\begin{Arguments}
\begin{ldescription}
\item[\code{x}] an object of class \code{"efp"}.
\item[\code{alt.boundary}] logical. If set to \code{TRUE} alternative boundaries
(instead of
the standard linear boundaries) will be used (for CUSUM
processes only).
\item[\code{functional}] indicates which functional should be applied to the
empirical fluctuation process.
\item[\code{...}] currently not used.
\end{ldescription}
\end{Arguments}
\begin{Details}\relax
The critical values for the MOSUM tests and the ME test are just
tabulated for confidence levels between 0.1 and 0.01, thus the p
value approximations will be poor for other p values. Similarly the
critical values for the maximum and mean squared Euclidian norm (\code{"maxL2"}
and \code{"meanL2"}) are tabulated for confidence levels between 0.2 and 0.005.\end{Details}
\begin{Value}
an object of class \code{"htest"} containing:
\begin{ldescription}
\item[\code{statistic}] the test statistic
\item[\code{p.value}] the corresponding p value
\item[\code{method}] a character string with the method used
\item[\code{data.name}] a character string with the data name
\end{ldescription}
\end{Value}
\begin{References}\relax
Brown R.L., Durbin J., Evans J.M. (1975), Techniques for
testing constancy of regression relationships over time, \emph{Journal of the
Royal Statistal Society}, B, \bold{37}, 149-163.

Chu C.-S., Hornik K., Kuan C.-M. (1995), MOSUM tests for parameter
constancy, \emph{Biometrika}, \bold{82}, 603-617.

Chu C.-S., Hornik K., Kuan C.-M. (1995), The moving-estimates test for
parameter stability, \emph{Econometric Theory}, \bold{11}, 669-720.

Kr�mer W., Ploberger W., Alt R. (1988), Testing for structural change in
dynamic models, \emph{Econometrica}, \bold{56}, 1355-1369.

Kuan C.-M., Hornik K. (1995), The generalized fluctuation test: A
unifying view, \emph{Econometric Reviews}, \bold{14}, 135 - 161.

Kuan C.-M., Chen (1994), Implementing the fluctuation and moving estimates
tests in dynamic econometric models, \emph{Economics Letters}, \bold{44},
235-239.

Ploberger W., Kr�mer W. (1992), The CUSUM Test with OLS Residuals,
\emph{Econometrica}, \bold{60}, 271-285.

Zeileis A. (2000), p Values and Alternative Boundaries for CUSUM Tests,
Working Paper 78, SFB "Adaptive Information Systems and Modelling in Economics
and Management Science", Vienna University of Economics,
\url{http://www.wu-wien.ac.at/am/wp00.htm#78}.\end{References}
\begin{SeeAlso}\relax
\code{\Link{efp}}, \code{\Link{plot.efp}}\end{SeeAlso}
\begin{Examples}
\begin{ExampleCode}
## Load dataset "nhtemp" with average yearly temperatures in New Haven
data(nhtemp)
## plot the data
plot(nhtemp)

## test the model null hypothesis that the average temperature remains
## constant over the years compute OLS-CUSUM fluctuation process
temp.cus <- efp(nhtemp ~ 1, type = "OLS-CUSUM")
## plot the process with alternative boundaries
plot(temp.cus, alpha = 0.01, alt.boundary = TRUE)
## and calculate the test statistic
sctest(temp.cus)

## compute moving estimates fluctuation process
temp.me <- efp(nhtemp ~ 1, type = "ME", h = 0.2)
## plot the process with functional = "max"
plot(temp.me)
## and perform the corresponding test
sctest(temp.me)
\end{ExampleCode}
\end{Examples}

