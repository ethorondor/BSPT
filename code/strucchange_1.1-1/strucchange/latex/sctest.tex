\Header{sctest.formula}{Structural Change Tests}
\alias{sctest}{sctest.formula}
\keyword{htest}{sctest.formula}
\begin{Description}\relax
Performs tests for structural change.\end{Description}
\begin{Usage}
\begin{verbatim}
## S3 method for class 'formula':
sctest(formula, type = <<see below>>, h = 0.15,
    alt.boundary = FALSE, functional = c("max", "range",
    "maxL2", "meanL2"), from = 0.15, to = NULL, point = 0.5,
    asymptotic = FALSE, data, ...)
\end{verbatim}
\end{Usage}
\begin{Arguments}
\begin{ldescription}
\item[\code{formula}] a formula describing the model to be tested.
\item[\code{type}] a character string specifying the structural change test that ist
to be performed. Besides the tests types described in \code{\Link{efp}}
and \code{\Link{sctest.Fstats}}. The Chow test and the Nyblom-Hansen test
can be performed by setting type to \code{"Chow"} or \code{"Nyblom-Hansen"},
respectively.
\item[\code{h}] numeric from interval (0,1) specifying the bandwidth. Determins the
size of the data window relative to sample size (for MOSUM and ME tests
only).
\item[\code{alt.boundary}] logical. If set to \code{TRUE} alternative boundaries
(instead of the standard linear boundaries) will be used (for CUSUM
processes only).
\item[\code{functional}] indicates which functional should be used to aggregate
the empirical fluctuation processes to a test statistic.
\item[\code{from, to}] numerics. If \code{from} is smaller than 1 they are
interpreted as percentages of data and by default \code{to} is taken to be
the 1 - \code{from}. F statistics will be calculated for the observations
\code{(n*from):(n*to)}, when \code{n} is the number of observations in the
model. If \code{from} is greater than 1 it is interpreted to be the index
and \code{to} defaults to \code{n - from}. (for F tests only)
\item[\code{point}] parameter of the Chow test for the potential change point.
Interpreted analogous to the \code{from} parameter. By
default taken to be \code{floor(n*0.5)} if \code{n} is the  number of
observations in the model.
\item[\code{asymptotic}] logical. If \code{TRUE} the asymptotic (chi-square)
distribution instead of the exact (F) distribution will be used to compute
the p value (for Chow test only).
\item[\code{data}] an optional data frame containing the variables in the model. By
default the variables are taken from the environment which
\code{sctest} is called from.
\item[\code{...}] further arguments passed to \code{\Link{efp}} or
\code{\Link{Fstats}}.
\end{ldescription}
\end{Arguments}
\begin{Details}\relax
\code{sctest.formula} is mainly a wrapper for \code{\Link{sctest.efp}}
and \code{\Link{sctest.Fstats}} as it fits an empirical fluctuation process
first or computes the F statistics respectively and subsequently performs the
corresponding test. The Chow test and the Nyblom-Hansen test are available explicitely here.\end{Details}
\begin{Value}
an object of class \code{"htest"} containing:
\begin{ldescription}
\item[\code{statistic}] the test statistic
\item[\code{p.value}] the corresponding p value
\item[\code{method}] a character string with the method used
\item[\code{data.name}] a character string with the data name
\end{ldescription}
\end{Value}
\begin{SeeAlso}\relax
\code{\Link{sctest.efp}}, \code{\Link{sctest.Fstats}}\end{SeeAlso}
\begin{Examples}
\begin{ExampleCode}
## Load dataset "nhtemp" with average yearly temperatures in New Haven
data(nhtemp)
## plot the data
plot(nhtemp)

## test the model null hypothesis that the average temperature remains
## constant over the years with the Standard CUSUM test
sctest(nhtemp ~ 1)
## with the Chow test (under the alternative that there is a change 1941)
sctest(nhtemp ~ 1, type = "Chow", point = c(1941,1))
\end{ExampleCode}
\end{Examples}

