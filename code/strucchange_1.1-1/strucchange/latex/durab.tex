\Header{durab}{US Labor Productivity}
\keyword{datasets}{durab}
\begin{Description}\relax
US labor productivity in the manufacturing/durables sector.\end{Description}
\begin{Usage}
\begin{verbatim}data(durab)\end{verbatim}
\end{Usage}
\begin{Format}\relax
\code{durab} is a multivariate monthly time series from 1947(3)
to 2001(4) with variables
\describe{
\item[y] growth rate of the Industrial Production Index to
average weekly labor hours in the manufacturing/durables sector,
\item[lag] lag 1 of the series \code{y},
}\end{Format}
\begin{Source}\relax
The data set is available from Bruce Hansen's homepage
\url{http://www.ssc.wisc.edu/~bhansen/}. For more information
see Hansen (2001).\end{Source}
\begin{References}\relax
Hansen B. (2001), The New Econometrics of Structural Change:
Dating Breaks in U.S. Labor Productivity,
\emph{Journal of Economic Perspectives}, \bold{15}, 117-128.

Zeileis A., Leisch F., Kleiber C., Hornik K. (2002), Monitoring
Structural Change in Dynamic Econometric Models,
Report 64, SFB "Adaptive Information Systems and Modelling in Economics
and Management Science", Vienna University of Economics,
\url{http://www.wu-wien.ac.at/am/reports.htm#78}.\end{References}
\begin{Examples}
\begin{ExampleCode}
data(durab)
## use AR(1) model as in Hansen (2001) and Zeileis et al. (2002)
durab.model <- y ~ lag

## historical tests
## OLS-based CUSUM process
ols <- efp(durab.model, data = durab, type = "OLS-CUSUM")
plot(ols)
## F statistics
fs <- Fstats(durab.model, data = durab, from = 0.1)
plot(fs)
## F statistics based on heteroskadisticy-consistent covariance matrix
fsHC <- Fstats(durab.model, data = durab, from = 0.1, cov.type = "HC")
plot(fsHC)

## monitoring
Durab <- window(durab, start=1964, end = c(1979, 12))
ols.efp <- efp(durab.model, type = "OLS-CUSUM", data = Durab)
newborder <- function(k) 1.5778*k/192
ols.mefp <- mefp(ols.efp, period=2)
ols.mefp2 <- mefp(ols.efp, border=newborder)
Durab <- window(durab, start=1964)
ols.mon <- monitor(ols.mefp)
ols.mon2 <- monitor(ols.mefp2)
plot(ols.mon)
lines(boundary(ols.mon2), col = 2)

## dating
bp <- breakpoints(durab.model, data = durab)
summary(bp)
plot(summary(bp))

plot(ols)
lines(breakpoints(bp, breaks = 1), col = 3)
lines(breakpoints(bp, breaks = 2), col = 4)
plot(fs)
lines(breakpoints(bp, breaks = 1), col = 3)
lines(breakpoints(bp, breaks = 2), col = 4)
\end{ExampleCode}
\end{Examples}

