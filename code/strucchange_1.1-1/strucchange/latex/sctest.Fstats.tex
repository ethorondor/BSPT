\Header{sctest.Fstats}{supF-, aveF- and expF-Test}
\keyword{htest}{sctest.Fstats}
\begin{Description}\relax
Performs the supF-, aveF- or expF-test\end{Description}
\begin{Usage}
\begin{verbatim}
## S3 method for class 'Fstats':
sctest(x, type = c("supF", "aveF", "expF"),
    asymptotic = FALSE, ...)
\end{verbatim}
\end{Usage}
\begin{Arguments}
\begin{ldescription}
\item[\code{x}] an object of class \code{"Fstats"}.
\item[\code{type}] a character string specifying which test will be performed.
\item[\code{asymptotic}] logical. Only necessary if \code{x} contains just a single F
statistic and type is \code{"supF"} or \code{"aveF"}. If then set to
\code{TRUE} the asymptotic (chi-square) distribution instead of the exact
(F) distribution will be used to compute the p value.
\item[\code{...}] currently not used.
\end{ldescription}
\end{Arguments}
\begin{Details}\relax
If \code{x} contains just a single F statistic and type is
\code{"supF"} or \code{"aveF"} the Chow test will be performed.

The original GAUSS code for computing the p values of the supF-, aveF- and
expF-test was written by Bruce Hansen and is available from
\url{http://www.ssc.wisc.edu/~bhansen/}. R port by Achim Zeileis.\end{Details}
\begin{Value}
an object of class \code{"htest"} containing:
\begin{ldescription}
\item[\code{statistic}] the test statistic
\item[\code{p.value}] the corresponding p value
\item[\code{method}] a character string with the method used
\item[\code{data.name}] a character string with the data name
\end{ldescription}
\end{Value}
\begin{References}\relax
Andrews D.W.K. (1993), Tests for parameter instability and structural
change with unknown change point, \emph{Econometrica}, \bold{61}, 821-856.

Andrews D.W.K., Ploberger W. (1994), Optimal tests when a nuisance parameter
is present only under the alternative, \emph{Econometrica}, \bold{62}, 1383-1414.

Hansen B. (1992), Tests for parameter instability in regressions with I(1)
processes, \emph{Journal of Business \& Economic Statistics}, \bold{10}, 321-335.

Hansen B. (1997), Approximate asymptotic p values for structural-change
tests, \emph{Journal of Business \& Economic Statistics}, \bold{15}, 60-67.\end{References}
\begin{SeeAlso}\relax
\code{\Link{Fstats}}, \code{\Link{plot.Fstats}}\end{SeeAlso}
\begin{Examples}
\begin{ExampleCode}
## Load dataset "nhtemp" with average yearly temperatures in New Haven
data(nhtemp)
## plot the data
plot(nhtemp)

## test the model null hypothesis that the average temperature remains
## constant over the years for potential break points between 1941
## (corresponds to from = 0.5) and 1962 (corresponds to to = 0.85)
## compute F statistics
fs <- Fstats(nhtemp ~ 1, from = 0.5, to = 0.85)
## plot the F statistics
plot(fs, alpha = 0.01)
## and the corresponding p values
plot(fs, pval = TRUE, alpha = 0.01)
## perform the aveF test
sctest(fs, type = "aveF")
\end{ExampleCode}
\end{Examples}

