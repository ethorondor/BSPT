\Header{plot.Fstats}{Plot F Statistics}
\alias{lines.Fstats}{plot.Fstats}
\keyword{hplot}{plot.Fstats}
\begin{Description}\relax
Plotting method for objects of class \code{"Fstats"}\end{Description}
\begin{Usage}
\begin{verbatim}
## S3 method for class 'Fstats':
plot(x, pval = FALSE, asymptotic = FALSE, alpha = 0.05,
    boundary = TRUE, aveF = FALSE, xlab = "Time", ylab = NULL,
    ylim = NULL, ...)
\end{verbatim}
\end{Usage}
\begin{Arguments}
\begin{ldescription}
\item[\code{x}] an object of class \code{"Fstats"}.
\item[\code{pval}] logical. If set to \code{TRUE} the corresponding p values instead
of the original F statistics will be plotted.
\item[\code{asymptotic}] logical. If set to \code{TRUE} the asymptotic (chi-square)
distribution instead of the exact (F) distribution will be used to compute
the p values (only if \code{pval} is \code{TRUE}).
\item[\code{alpha}] numeric from interval (0,1) indicating the confidence level for
which the boundary of the supF test will be computed.
\item[\code{boundary}] logical. If set to \code{FALSE} the boundary will be computed
but not plotted.
\item[\code{aveF}] logical. If set to \code{TRUE} the boundary of the aveF test will
be plotted. As this is a boundary for the mean of the F statistics rather
than for the F statistics themselves a dashed line for the mean of the F
statistics will also be plotted.
\item[\code{xlab, ylab, ylim, ...}] high-level \code{\Link{plot}} function parameters.
\end{ldescription}
\end{Arguments}
\begin{References}\relax
Andrews D.W.K. (1993), Tests for parameter instability and structural
change with unknown change point, \emph{Econometrica}, \bold{61}, 821-856.

Hansen B. (1992), Tests for parameter instability in regressions with I(1)
processes, \emph{Journal of Business \& Economic Statistics}, \bold{10}, 321-335.

Hansen B. (1997), Approximate asymptotic p values for structural-change
tests, \emph{Journal of Business \& Economic Statistics}, \bold{15}, 60-67.\end{References}
\begin{SeeAlso}\relax
\code{\Link{Fstats}}, \code{\Link{boundary.Fstats}},
\code{\Link{sctest.Fstats}}\end{SeeAlso}
\begin{Examples}
\begin{ExampleCode}
## Load dataset "nhtemp" with average yearly temperatures in New Haven
data(nhtemp)
## plot the data
plot(nhtemp)

## test the model null hypothesis that the average temperature remains
## constant over the years for potential break points between 1941
## (corresponds to from = 0.5) and 1962 (corresponds to to = 0.85)
## compute F statistics
fs <- Fstats(nhtemp ~ 1, from = 0.5, to = 0.85)
## plot the F statistics
plot(fs, alpha = 0.01)
## and the corresponding p values
plot(fs, pval = TRUE, alpha = 0.01)
## perform the aveF test
sctest(fs, type = "aveF")
\end{ExampleCode}
\end{Examples}

